\begin{tikzpicture}



% http://www.texample.net/tikz/examples/turing-machine-2/

 \edef\h{0.7cm}
 \edef\w{0.7cm}
 \pgfmathtruncatemacro\wr{\w+0.4cm}

 \def\len{14}

 \tikzstyle{elem}=[draw,minimum height=\h, inner sep=2pt]

% http://tex.stackexchange.com/questions/19626/how-to-solve-the-10-09999-rounding-problem-with-pgfmath

 \pgfmathtruncatemacro\middle{\len/2+1}
 \pgfmathtruncatemacro\middleL{\middle-2}
 \pgfmathtruncatemacro\middleR{\len-\middle-2}


% \def\name{C}

 \foreach \y/\name in {0/T,1/D,2/G,3/P,4/R}
  {
   \begin{scope}[ yshift=-\y cm, start chain=going right
                , node distance=-0.25mm
                ]
    
    \foreach \i in {1,...,\middleL}
     \node[on chain, elem, minimum width=\w] (\name-\i) { $\mathrm{\name}_{\i}$ };
     
    \node[on chain, elem, minimum width=\w] {$\dots$};
    \node[on chain, elem, minimum width=\w] (\name-i) {$\mathrm{\name}_i$};
    \node[on chain, elem, minimum width=\w] {$\dots$};

    \foreach \i in {\middleR,...,1}
     \node[on chain, elem, minimum width=\wr] { $\mathrm{\name}_{n-\i}$ };

    \node[on chain, elem, minimum width=\w] (\name-n) { $\mathrm{\name}_n$ };
   \end{scope}
  }

 \node[ draw, fit={(T-i) (R-i)}
      , label=below:$\mathrm{Class}_i$
  ] (Ci) {};

\draw [ decorate,decoration={brace,amplitude=10pt}]
  ($(T-1.north west) + (0,0.3cm)$) -- ($(T-n.north east) + (0,0.3)$)
  node[midway, yshift=0.25cm, label=above:All classes for all groups];

\end{tikzpicture}